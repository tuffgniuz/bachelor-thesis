\section{Theoretical Foundation}

\subsection{Defining Information Security}
It is important to understand what exactly is meant when discussing security. Therefore, we have to properly define it. For instance, the terms cybersecurity and information security are commonly used indistinguishably. The Cambridge Dictionary defines information security as \say{methods used to prevent electronic information from being illegally obtained or used}, and defines cybersecurity as \say{things that are done to protect a person, organization, or country and their computer information against crime or attacks carried out using the internet}. The definition for cybersecurity describes a much larger scope for security. 

Solms et al \cite{von_Solms_2013} supports this argument and reasons that information security is solely about securing the information, generally referred to as the asset, from potential threats posed by inherent vulnerabilities. Furthermore, they outlined that cybersecurity goes beyond protecting assets. Cybersecurity includes the protection of those that function in cyberspace in addition to any of their assets that can be attained through cyberspace. Although the definition of information security and cybersecurity overlap each other, the latter is much more extensive in its definition. Overall, security is about securing assets against the most likely forms of attacks, to the best ability \cite{andress2014the}.

% Information security can be defined as the protection of information from potential abuse subsequent from various threats and vulnerabilities \cite{von_Solms_2013}. In a general sense, security means protecting our data and systems assets from whoever who intends to misuse it. It includes several aspects of business, involving financial controls, human resources and protection of the physical environment, as well as health and safety measures \cite{zinatullin2016the}. Security strives to secure ourselves against the most likely forms of attack, to the best ability.


\subsection{Challenges in Information Security}\label{sec:challenges-in-is}
Computers have evolved drastically from the mechanical calculating machines they once were when first introduced in the 19th century. Today, computers are powerful machines that allows us to surf the internet, run games, and stream multimedia. Everything involves system and software technology that constantly is processing information. Important and sensitive information are controlled, managed or either form part of the cyberspace. Likewise, important and sensitive information can be exploited as well in this space \cite{Li_2021}. Moreover, it has become very simple to stage an attack. A sophisticated attack can be constructed that affect millions of computers worldwide by an attacker with tutorials found online together with their own knowledge and resources. 

In addition, many organizations perceive security as a hindrance to productivity. It is not uncommon for discussions concerning security to be avoided  among business leaders and IT personnel. Errors made by developers might be costly and might endanger everyone who trust the software they build. Though the stakes in security are high, developers perceive security as a secondary concern \cite{Lopez_2019}. 

Even so, security can be complicated without the appropriate approach. As the level of security is increased, the level of productivity is usually decreased. Therefore, the role of a security plan is to find the balance between protection, usability, and cost. Moreover, in what manner the level of security relates to the value of the item being secured needs to be taken into consideration, when an asset, system, or environment is secured. 

Knowledge on how to stay secure changes at a much slower pace in contrast to the increasingly accelerated rate technology changes. As a consequence, security does not always keep up with the changes. At the same time, gaining a good understanding of the fundamentals of information security will provide the foundation needed to manage changes as they come along.

\subsection{Secure By Design}
The term secure can be defined as \say{freedom from risk and the threat of change for the worse}. From the software engineers perspective security is about engineering software such that assets are free from risk and the threat of change for the worse, or at least to the best ability. Secure programming is about designing and implementing software with the minimal amount of vulnerabilities that an attacker can exploit \cite{helfrich2019security}. However, modern software is complex and fragile. Despite professional engineers being capable of testing and debugging code, security is a different issue, because insecure code generally works without issues, given no attacker is exploiting the code.

\subsection{Software Security Requirement}
