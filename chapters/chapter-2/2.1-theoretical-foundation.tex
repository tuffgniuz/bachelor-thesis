% In this chapter the challenges faced in information security is discussed, followed by the common exploits described in OWASP Top 10 and the fundamental concepts of information security. Lastly, this section will cover how security can fit into the SDLC.

This chapter examines the theoretical foundation and frameworks which will be used or support the design of solution for the problem.

\section{Theoretical Foundation}
In this section, the relevant theories are outlined and presented comprehensively and in accordance with the problem.

\subsection{Information Theory}
Information security can be defined as the protection of information from potential abuse subsequent from various threats and vulnerabilities \cite{von_Solms_2013}. Generally speaking, security means protecting our data and systems assets from whoever who intends to misuse it. Security strives to secure ourselves against the most likely forms of attack, to the best ability.
\subsection{Challenges in Information Security}\label{sec:challenges-in-is}
Computers have evolved drastically from the mechanical calculating machines designed and built to solve the increasingly complex number-crunching challenges during the 19th century \cite{history_of_computers}. Today, computers are powerful machines that allows us to surf the internet, run games, and stream multimedia. Everything involves system and software technology that constantly is processing information. Important and sensitive information are controlled, managed or either form part of the cyberspace. Likewise, important and sensitive information can be exploited as well in this space \cite{Li_2021}. Moreover, it has become very simple to stage an attack due to several factors including:
\begin{itemize}
    \item End users are prone to clicking website URLs and launching attachments in emails
    \item Buying DIY kits from hackers to develop malware
    \item Processing power 
    \item Getting the latest exploits from underground services
\end{itemize}
A sophisticated attack can be constructed that affect millions of computers worldwide by an attacker that take these, chain them together with tutorials found online together with their own knowledge and resources. Alas, the modern computer is not developed with security in mind. Since the very beginning, an element of trust has been inbred into computers. Consequently, security still too often is an afterthought, causing an epidemic of vulnerable information systems \cite{death2017information}.

In addition, many organizations perceive security as a hindrance to productivity. It is not uncommon for discussions concerning security to be avoided  among business leaders and IT personnel. Errors made by developers might be costly and might endanger everyone who trust the software they build. Though the stakes in security are high, developers perceive security as a secondary concern \cite{Lopez_2019}. 

Even so, security can be complicated without the appropriate approach. As the level of security is increased, the level of productivity is usually decreased. Therefore, the role of a security plan is to find the balance between protection, usability, and cost. Moreover, in what manner the level of security relates to the value of the item being secured needs to be taken into consideration, when an asset, system, or environment is secured. In Section \ref{sec:opsec} the thesis will go in depth over the important policies and concepts that cover information security.

Knowledge on how to stay secure changes at a much slower pace in contrast to the increasingly accelerated rate technology changes. As a consequence, security does not always keep up with the changes. At the same time, gaining a good understanding of the fundamentals of information security will provide the foundation needed to manage changes as they come along.
