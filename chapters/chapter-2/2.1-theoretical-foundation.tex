\section{Theoretical Foundation}

\subsection{Defining Information Security}
Information security can be defined as the protection of information from potential abuse subsequent from various threats and vulnerabilities \cite{von_Solms_2013}. In a general sense, security means protecting our data and systems assets from whoever who intends to misuse it. It includes several aspects of business, involving financial controls, human resources and protection of the physical environment, as well as health and safety measures \cite{zinatullin2016the}. Security strives to secure ourselves against the most likely forms of attack, to the best ability.


\subsection{Challenges in Information Security}\label{sec:challenges-in-is}
Computers have evolved drastically from the mechanical calculating machines they once were when first introduced in the 19th century. Today, computers are powerful machines that allows us to surf the internet, run games, and stream multimedia. Everything involves system and software technology that constantly is processing information. Important and sensitive information are controlled, managed or either form part of the cyberspace. Likewise, important and sensitive information can be exploited as well in this space \cite{Li_2021}. Moreover, it has become very simple to stage an attack. A sophisticated attack can be constructed that affect millions of computers worldwide by an attacker with tutorials found online together with their own knowledge and resources. 

In addition, many organizations perceive security as a hindrance to productivity. It is not uncommon for discussions concerning security to be avoided  among business leaders and IT personnel. Errors made by developers might be costly and might endanger everyone who trust the software they build. Though the stakes in security are high, developers perceive security as a secondary concern \cite{Lopez_2019}. 

Even so, security can be complicated without the appropriate approach. As the level of security is increased, the level of productivity is usually decreased. Therefore, the role of a security plan is to find the balance between protection, usability, and cost. Moreover, in what manner the level of security relates to the value of the item being secured needs to be taken into consideration, when an asset, system, or environment is secured. In Section \ref{sec:opsec} the thesis will go in depth over the important policies and concepts that cover information security.

Knowledge on how to stay secure changes at a much slower pace in contrast to the increasingly accelerated rate technology changes. As a consequence, security does not always keep up with the changes. At the same time, gaining a good understanding of the fundamentals of information security will provide the foundation needed to manage changes as they come along.
