The final chapter discusses the conclusion of the thesis and any further recommendation based on the findings.

\section{Conclusion}
The study set out to implement SKF in an agile SDLC. As a case study, the development of a full stack application was conducted. This included the development of the API and the clientside. To ensure security activities were performed during the SDLC, SKF was used to do security requirements, security design, secure code implementation, and security testing. In this study SKF was used to achieve MASVS-L1 wich is the bare minimum that all mobile applications should adhere too. Specifically, security requirements were gathered for the development of the authentication and user session management feature of CheFeed. Chapter 5 showed the code implementation based on the security requirements and design from chapter 4. Lastly, security tests were performed in the form of manual tests and static analysis tools. SKF is great in the functional part of the SDLC. However, without a proper and clear process ahead SKF by itself may not suffice. 

\section{Recommendation}
SKF should be used in complement with processes such as OWASP SAMM or Microsoft SDL. Both provide clear security activities throughout the whole SDLC. An important part that was missing during this study, was introducing security training during the SDLC. Without the proper security training, security is difficult to understand and even more difficult to implement. Fortunately, a training platform is provided by SKF with free courses. Future studies might benefit by developing a secure SDLC where SKF is combined with OWASP SAMM or Microsoft SDL.


