This chapter will evaluate the process of developing mobile applications with React Native and implementing security requirements based on SKF.

\section{Mobile Application Development}
React Native has allowed for a relative easy and fast development time. Furthermore, it is easy to learn when the developer is already familiar with JavaScript and React DOM. Although the introduction of JavaScript on mobile platforms may cause security concern, the development might outweigh those concerns. However, if a proper security risk assessment and threat modeling is done, a development team can gather proper security requirements and still deliver secure mobile application no matter the programming language or the technology stack being used. In addition, security standards such as the ASVS and MASVS provided by OWASP ensures that users should do proper risk assessment or threat modeling. However, being the frontend developer and security requirement engineer has proven to be difficult and time-consuming. Though gathering the security requirements is relatively easy due to SKFs security expert system, also being the frontend developer is a difficult task. In practice, the security engineer or security champion should delegate the development to the software developers.

\section{Designing Secure Applications}
OWASP is a great source for security tools. They have provided great security standards such as the ASVS and MASVS, as well as security testing guides such as the MSTG. SKF benefits from these indivual OWASP project and provides an easy-to-use web application to access and generate security requirements that can be configured for each development sprint during the SDLC. Though the ASVS and MASVS are great documents, they are designed such that an organization is free in the usage of their standards. SKF makes it easier to use those standards.
