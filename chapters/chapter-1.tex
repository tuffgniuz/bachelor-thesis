\section{Background}
Throughout the last ten years, mass cyberattacks against major organizations have amplified. Security breaches are the most prominent cause for attacks being allowed to happen. Although different types of organizations become victim of cyberattacks, an analysis of data breaches experienced in multiple organizations established that medical organizations and BSOs are the least prepared against attacks \cite{Hammouchi_2019}. The latest reports confirm that vulnerabilities continue to rise as a result of the pandemic, and almost half of all businesses have been confronted with cyberattacks since companies are settling into the new normal \cite{Weir_2021}. Cyberattacks have grown in frequency and severity since the pandemic. Lallie, Harjinder Singh et al observed that there appears to be a loose correlation between the announcement and a corresponding cyber-attack campaign that utilizes the event as a hook to increase the likelihood of success \cite{Lallie_2021}. Though solutions to counter cyberattacks exist, guidance on implementing software security is needed.

Security breaches are caused by security vulnerabilities in source code introduced by software developers when creating software, and therefore developers are often blamed for vulnerabilities \cite{Assal_2019}. However, application security is primarily performed by security experts causing a separation between security and development. As a result, the probability of insecure software is increased \cite{Thomas_2018}.

Writing secure code therefore is critical with the prevalence of security vulnerabilities. To achieve this, developers need to be aware of the potential vulnerabilities they might introduce when developing software features and understand how to mitigate them. Still, the knowledge and skills to produce secure software are lacking and often are not taught in computing curricula despite the existence of secure coding guidelines \cite{Tabassum_2018} \cite{yu2011teaching} \cite{Espinha_Gasiba_2021} \cite{blindspot_2018}.

Such secure code guidelines are provided by the Open Web application Security Project (OWASP). Security standards such as the OWASP Application Security Verification Standard (ASVS) and the Mobile Application Security Verification Standard (MASVS) exist to guide organizations and developers to produce secure software through categorized security controls that can be implemented during the development lifecycle. Furthermore, OWASP has cheat sheets and a series of security testing guides that provides test cases that verifies the security controls put in place. Organizations that are serious about security, should apply security standards and security testing guides during the Software Development Life Cycle (SDLC). The SDLC is a process used to plan, define requirements, design, implement, test and deploy software. It ensures that software development is done reliable, cost-effective in a given time window. Implementing security in the SDLC is the first step in assuring secure software is produced. However, developers might not find it easy to work with such guidelines \cite{Acar_2017}.

The OWASP Security Knowledge Framework (SKF) is security expert system that uses secure code guidelines such as the OWASP ASVS and OWASP MASVS to assist developers during the SDLC. To address security during the SDLC, this study will implement a secure SDLC (SSDLC) with SKF for the development of CheFeed, a full stack mobile application developed for this thesis as group. CheFeed is a social recipe sharing application that also applies sentiment analysis on recipe reviews. Though CheFeed is minimal in its functionality, features such as user authentication and user session management needs to ensure it follows security standards. Thus, CheFeed is an appropriate candidate for implementing a SSDLC. 

\section{Aims and Benefits}

\subsection{Aim}
Security vulnerabilities found in applications introduced by software developers are the cause of many security breaches and data theft. To reduce the introduction of security vulnerabilities security must be addressed during software development. The MASVS has categorized security controls that can be used as guidance to develop secure software. In addition, it maps security controls to test cases that can be found in the OWASP Mobile Security Testing Guide (MSTG). However, implementing security standards such as the MASVS might be a difficult task. Without a proper background in security, a developer might not know where to start and understand what security controls are relevant to the project. Therefore, this study aims to implement SKF in an agile SSDLC where security activies are performed based on MASVS security controls defined for feature sprints.   

%Furthermore, the study presents SKF as a security expert during the development lifecycle where it ensures security activities are performed for CheFeed.

\subsection{Benefits}
This study will provide insights into the process of developing software in an agile SDLC where security is addressed from the beginning. It provides a clear presentation of the benefits of using SKF to configure feature sprints where security should be in place. Students and software developers may benefit from understanding how secure code guidelines can be used. Furthermore, the implementation of SKF for a secure SDLC could serve as a tool for future studies to build upon the process presented in this study.

\section{Scope}
The MASVS and other security standards understands that not all applications are equal in terms of its potential for security vulnerabilities. Thus, several security verification levels exist. The scope of this study is limited to MASVS \emph{Level 1} (MASVS-L1). MASVS-L1 ensures that mobile applications cohere to the best security practices concerning code quality, handling of sensitive data, and interaction with the mobile environment. Security activities are performed during the \emph{requirements, design, implementation}, and \emph{testing} phases of the SDLC.

CheFeed is a full stack mobile application that has been developed as a final group project for our undergraduate thesis. The overall scope for the development of the project involves the development of the REST API, the development of a sentiment analysis model that will be served on the REST API processing recipe reviews from users, and the development of the client side. The focus of each contributor outside this study is delegated as follows:

\begin{itemize}
    \item The development of the API and database design by Stephanus Jovan Novarian in his thesis \say{CheFeed: Development of CRUD backend services on recipes}.
    \item The development of a sentiment analysis model which uses recipe reviews as its input by Ikshan Maulana in his thesis \say{CheFeed - The Implementation of different RNN Architectures}.
\end{itemize}

