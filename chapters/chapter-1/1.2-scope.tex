\section{Scope}
This study covers the implementation of SKF for developing secure mobile applications. SKF provides three security maturity levels from ASVS and MASVS. This study limits itself to achieve level 1 for security for the features users and authentication. In practice, a security risk assessment or threat modeling should be added to a secure software development cycle. Thefore, the thesis will limit itself to: 

\begin{itemize}
    \item gathering security requirements with SKF for chefeed user authentication and session managements
    \item development of the authentication API
    \item  development of the client side
    \item security verification
\end{itemize}

% The thesis focuses on fundamental security concepts and principles. They will be applied by developing the CHEFEED application. CHEFEED will have an API and a hybrid native client. The application will go through a secure Software Development Life Cycle (SDLC). An important principle in information security is to find a good balance between security and productivity. Thus, a security maturity model and a standard for writing secure code will guide the SDLC. Adding security into the SDLC will allow to have secure code by design, instead of having security as an afterthought. Furthermore, security tests will be performed in the form of penetration tests and security code reviews.

CHEFEED is a group effort and the scope and responsibilities outside this thesis are described in Table \ref{tab:member-scope}.

\begin{table}[!h]
    \centering
    \caption{Other member scope overview}
    \label{tab:member-scope}
    \begin{tabular}{|l|p{10em}|p{12em}|}
        \hline
        \textbf{Member} & \textbf{Scope} & \textbf{Thesis} \\
        \hline
        Ikhsan Maulana & Sentiment Analysis, back-end development & Sentiment Analysis on Food Reviews \\
        \hline
        Stephanus Jovan Novarian & SDLC, back-end development & Implement Agile Software Development Life Cycle on CHEFEED \\
        \hline
    \end{tabular}
\end{table}
