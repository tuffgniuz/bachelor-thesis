\section{Background}
Are software developers responsible for the cyberattacks that has increased over the last decade? An analysis of over 9000 data breaches since 2005 have revealed that data breaches have contributed to the loss of 11,5 billion individual records with major financial and technical impact \cite{Hammouchi_2019}. Countless attacks are allowed to happen as a consequence of software vulnerabilities that leave web applications, web servers, or websites exposed. Software developers increase the risk of a security vulnerability each time a new feature is added to an application, albeit many can be mitigated with a secure-by-design approach. However, software developers need to have adequate knowledge on how to mitigate common security vulnerabilities. Ergo the knowledge and skill to develop secure software needs to be taught to software developers, yet is commonly overlooked \cite{Tabassum_2018}.

% TODO: change the parapgraph below into practices done to discuss developer software security activity

% Several approaches to educate students on secure programming are commonly considered. The first, adding it as a material to existing classes. However, if the principles and practices are introduced as extra material in existing classes, the skill is atrophied, considering that whether the students write secure programs is not being examined. The second, adding a separate class that covers secure programming in an already packed curriculum. A third approach is the introduction of a "secure programming clinic" that facilitates students on the principles and practices of writing secure software \cite{Bishop_2017}. The University of North Carolina at Charlotta introduced their own tool Educational Security in the IDE, which provides students with security warnings on assignment code for integrating secure programming education. The paper states that students awareness and knowledge on secure programming increases through the usage of ESIDE \cite{Tabassum_2018}.

% TODO: Introduce the agile software development life cycle
% TODO: introduce chefeed and reasoning for developing a recipe app
% TODO: introduce sentiment analysis

For this thesis a hybrid mobile application will be developed using FastAPI, a python framework to develop the Application Programming Interface (API), and React Native, a JavaScript library to develop User Interfaces. The mobile application will be a social recipe sharing called CheFeed.

A well known source for best practices in software security is the Open Web Application Security Project (OWASP). OWASP is a nonprofit foundation that concerns itself on improving the security of software, and has over 250 local chapters around the globe. It includes community-led open source software projects \cite{owasp_about}. This thesis proposes to implement the Security Knowledge Framework (SKF) into an agile Software Development Life Cycle (SDLC). SKFs objective is to train developers and teams in writing secure code using the Application Security Verification Standard (ASVS) and Mobile Application Security Verification Standard (MASVS).

% To learn secure programming practices, this thesis proposes to integrate the Security Knowledge Framework into the Software Development Life Cycle with the development of a social recipe sharing app, CheFEED. SKF is an open-source flagship project from the Open Web Application Security Project. It utilizes the OWASP Application Security Verification Standard to train developers in writing secure code by design. The ASVS provides a framework of security requirements and controls for designing, developing and testing modern web applications and web services. In addition they provide the Mobile Application Security Verification Standard that focuses security requirements and controls for mobile applications. SKF in addition has developed a security expert system that generates these security requirements for a sprint during the SDLC.

% The purpose of this thesis is to examine whether SKF is a good source for developers to learn about secure software development, without a security champion in the team. The thesis is conducted as a case study to see if skill and abilities in secure programming is enhanced through the development of the CheFEED application. The features of CheFEED will form the foundation to generate the security requirements needed to develop a secure application.

