% Introduction
% This chapter aims at understanding the problem.

\section{Challenges in Software Security}

\section{Existing Solutions}

\subsection{The OWASP Top Ten}

\subsection{The OWASP Application Security Verification Standard}

\subsection{The OWASP Mobile Application Security Verification Standard and the Mobile Security Testing Guide}
The MASVS similar to the ASVS provides a security standard to design, develop and test secure mobile apps. Although smartphones are small computers running on an operating system with mobile apps, the security requirements are not necessary similar. Data such as our personal information, pictures, recordings, notes, account data, business information, location and much more are stored on mobile apps. Therefore, mobile security is centered around data protection \cite{owasp_masvs}.
\subsubsection{MASVS Verification Levels}
Unlike the ASVS The AppSec model for mobile applications defines two security verification levels. However, in addition to the two security verification levels the MASVS adds a set of reverse engineering resiliency requirements.

\begin{itemize}
    \item \textbf{MASVS-L1}: Standard Security
    \item \textbf{MASVS-L2}: Defense-in-Depth
    \item \textbf{MASVS-R}: Resiliency Against Reverse Engineering and Tampering
\end{itemize}
The MASVS Level 1 (L1) achieves basic requirements and complies to mobile application security best practices. Testing is required to validate the security controls. All mobile applications are recommended to implement L1.

\subsection{The OWASP Security Knowledge Framework}
OWASP SKF aims to train and guide software developers in building secure applications, by design. It is an open source web application that can be accessed online or be hosted locally with docker. It can be used by developers to gather security requirements, design secure code, and verify security tests. Furthermore, it includes features such as:
\begin{itemize}
    \item an extensivce library code examples
    \item ASVS and MASVS security checklist
    \item a platform to train secure coding and hacking skills
    \item knowledgebase items that are mapped to security items
\end{itemize}
Furthermore, SKF is supported by companies such as ING, Microsoft and Google.

% Proposed Solution
\section{Proposed Solution}
This thesis proposes SKF to design secure mobile applications. SKF allows security to be integrated into the SDLC for specific sprints. However, it is not an appropriate solution for security risk assessment or threat modeling. Therefore, this thesis concerns only with generating security requirements through SKF and how to work with those set of requirements. This section will discuss the funtional requirements of CheFeed, and what assets needs to be secured and their security requirements generated by SKF.

\subsection{User Stories}
The user stories in Table \ref{} describes the functional requirements for CheFeed.

\begin{table}
    \centering
    \caption{User Stories}
    \label{}
    \begin{tabulary}{1.0\textwidth}{|L|L|}
        \hline
          
        \hline
    \end{tabulary}
\end{table}

\subsection{Security Requirements}\label{subsec:sec-req}
Users are essential and important assets of CheFeed. Therefore, it is critical that they can securely store their credentials on their phone. As described in Table \ref{}, users register with their email and password and need to authenticate with those credentials. Without proper security in place, malicious users would be able to retrieve the end users credentials and wreak havoc on their accounts. 

As seen in the latest installment of the Top 10 Mobile Risk described in Table \ref{}, \say{Insecure Authentication} ranks number four in the list in the latest OWASP Top Ten, \say{Broken Access Control} ranks as the top security risk in web applications. The security requirements for CheFeed are all for the type mobile applications using maturity level 1. The requirements form the foundation to design and implement the security features into CheFeed.

\subsubsection{Users and Authentication}
This subsection describes the security requirements as they are generated by SKFs security expert system for the user and authentication sprint of CheFeeds SDLC. The configuration for this sprint is described in Table \ref{tab:users-and-authentication-config}. To ensure the best security requirements for CheFeeds user authentication and session management, security controls from three different categories from the MASVS we're gathered. A summary of the generated requirements can be seen in \ref{tab:chefeeds-sec-req-summary}. In total CheFeed has a total of thirteens security requirements based on the sprint configurations for this sprint.

\begin{table}
    \centering
    \caption{Users and Authentication Sprint Configurations}
    \label{tab:users-and-authentication-config}
    \begin{tabulary}{1.0\textwidth}{|L|L|L|}
        \hline
        \textbf{Category} & \textbf{Configurations} & \textbf{Option} \\
        \hline
        Authentication and Session Management Requirements & Abnormal account activity controls. & \textit{No} \\
                                                           & Authentication/Authorization requirements and lifecyle controls & \textit{Yes} \\
                                                           & Session management requirements and lifecyle & \textit{Yes} \\
                                                           \hline
        Data Storage and Privacy Requirements & Does your application keep sensitive data and share it with others? & \textit{Yes} \\
                                              & Does your Application keep sensitive data on the client side? & \textit{Yes} \\
                                              & Does your application need to provide privacy angainst prying eyes? & \textit{Yes} \\
                                              & Does your application process personal identifiable information? & \textit{Yes} \\
                                              \hline
        Cryptography Requirements & Does your application need to use symmetric encryption & \text{Yes} \\
                                  & Security best practice requirement & \textit{Yes} \\
                                  & Does your application need to generate secure random numbers for encryption & \text{Yes} \\
        \hline
    \end{tabulary}
\end{table}

\begin{table}
    \caption{Users and Authentication Sprint Configuration Summary}
    \label{tab:chefeeds-sec-req-summary}
    \begin{tabulary}{1.0\textwidth}{|L|L|}
        \hline
        \textbf{#} & \textbf{Description} \\
        \hline
        2.1 & System credential storage facilities need to be used to store sensitive data, such as PII, user credentials or cryptographic keys. \\
        \hline
        2.2 & No sensitive data should be stored outside the app container or system credential storage facilities. \\
        \hline
        2.3. & No sensitive data is written to application logs. \\
        \hline
        2.5. & The keyboard cache is disabled on text inputs that process sensitive data. \\
        \hline
        2.6 & No sensitive data is exposed via IPC mechanisms \\
        \hline
        2.7 & No sensitive data, such as passwords or pins, is exposed through the user interface. \\
        \hline
        4.1 & If the app provides users access to a remote service, some form of authentication, such as username/password authentication, is performed at the remote endpoint. \\
        \hline
        4.2 & If stateful session management is used, the remote endpoint uses randomly generated session identifiers to authenticate client requests without sending the user's credentials. \\
        \hline
        4.3 & If stateless token-based authentication is used, the server provides a token that has been signed using a secure algorithm. \\
        \hline
        4.4 & The remote endpoint terminates the existing session when the user logs out. \\
        \hline
        4.5 & A password policy exists and is enforced at the remote endpoint. \\
        \hline
        4.7 & Sessions are invalidated at the remote endpoint after a predefined period of inactivity and access tokens expire. \\
        \hline
        4.12 & Authorization models should be defined and enforced at the remote endpoint. \\
        \hline
    \end{tabulary}
\end{table}
