The final chapter discusses the conclusion of the thesis and any further recommendation based on the findings.

\section{Conclusion}
A security first approach to software development is important in today's modern world. However, not all application are processing the same types of assets. A banking applications might require a lot more security requirements than a simple to do app. Therefore, software developers need to be properly trained to understand what security risks their application might face and how to mitigate against different attacks. Fortunately security standards such as the ASVS and MASVS together with security testing guides such as the MSTG exist to assist developers in developing secure applications by design. SKF allows developers to easily gather security requirements from both the ASVS and MASVS and configure them for each feature sprint during the SDLC. In addition, SKF provides links to the MSTG to quickly access information on security testing for mobile applications. Furthermore, SKF includes features such as security labs and courses to train teams and individual developers who need or want to level up their security knowledge. Organization and individual developers would benefit from integrating the framework into their SDLC.

\section{Recommendation}
SKF is a great tool to easily gather security requirements for development projects. However, it can not be used to perform a security risk assessment or threat modeling. Therefore, it is recommended to combine the framework with projects such as OWASP SAMM or the Microsoft SDL. Both introduce security throughout all phases of the development process, where SKF can provide guidance in the training, requirements, design, implementation and verification phase. A combination of such security assurance processes together with SKF can therefore increase the security of any given application, either be that a web or mobile application.
