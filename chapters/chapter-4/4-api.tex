\section{API Design}
The CheFeed API is organized around REST. This section describes how the API is designed around the resources.

\subsection{Core Resources}
\begin{itemize}
    \item Recipes
    \item Categories
    \item Ingredients
    \item Bookmarks
    \item Reviews (comments)
\end{itemize}

\subsubsection{Recipes}
This object represents a Recipe object that belongs to a User object. Reading recipes does not require authentication, thus users without accounts are authorized to view recipes. Table \ref{tab:recipe-api} describes all the resources related to the Recipe object and what HTTP request methods are available.

\lstset{language=json}
\lstset{caption=The Recipe example object}
\begin{lstlisting}
{
    "id": "",
    "title": "Rendang",
    "description": "The world's tastiest beef stew",
    "cooking_time": 3,
    "image": "path/to/image",
    "video": "path/to/video",
    "created_at": "2022-08-02",
    "updated_at": ""
}
\end{lstlisting}

\subfile{../../tables/api-recipes}

\subsubsection{Categories}
This object represents a Category object and the resources are described in Table \ref{tab:category-api}.

\subfile{../../tables/api-categories}

\subsubsection{Ingredients}
This object represents an Ingredient object and the endpoints for this resouce are described in Table \ref{tab:ingredient-api}.

\subfile{../../tables/api-ingredients}

\subsubsection{Bookmarks}
This objects represents a Bookmark object and the endpoints for this resource are described in Table \ref{tab:bookmark-api}.

\subfile{../../tables/api-bookmarks}

\subsubsection{Reviews}
This object represents a Review object and the endpoints for this resource are described in Table \ref{tab:review-api}. The Review object can also be referred to as the Comment object as they both describe the same concept within the application.

\subfile{../../tables/api-reviews}

\subsection{Authentication}
Chefeed handles its authentication mechanism and user management through the FastAPI Users library, a ready-to-use library for FastAPI.

\subsection{Errors}

\subsection{Versioning}
Changes are inevitable in software development, and it must be managed well. Without proper management, changes can threaten the client integrity. APIs only are required to up-versioned when a breaking change is introduced. For instance changes in the format of the response data for one or more requests, a change in the request or response type, or whenever any part of the API is removed. Chefeeds API follows the versioning method similar to that of Stripe.

