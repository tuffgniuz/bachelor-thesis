This chapter will discuss the solution design to develop CheFeed with a security first approach. First the overall software architecture is described followed by the requirements' specification through user stories and acceptance criteria.

\section{Software Architecture}

\subsection{Authentication Flow}

\subsection{Authorization Flow}

\section{Requirement Specification}
In this section the requirements are described in the form of user stories and their correlated acceptance criteria. The acceptance criteria will be generated by SKF and validate the user stories. Table \ref{tab:chefeed-feature-list} provides an overview of all features for CheFeed and which ones require security verifications. 

\begin{table}[!h]
    \centering
    \caption{CheFeed feature descriptions}
    \label{tab:chefeed-feature-list}
    \begin{tabulary}{1.0\textwidth}{|L|L|L|C|}
        \hline
        \texbf{#} & \textbf{Epic} & \textbf{Description} & \textbf{Require Security Verification} \\
        \hline
        1 & Authentication \& Authorization & Create new user account & Yes \\
        \hline
        2 & Authentication \& Authorization & Login with created user credential  & Yes \\
        \hline
        3 & Authentication \& Authorizaation & Logout & Yes \\
        \hline
        4 & Recipes & View all recipes & No \\
        \hline
        5 & Recipes & View single recipe & No \\
        \hline
        6 & Recipes & Create recipe & Yes \\
        \hline
        7 & Recipes & Delete recipe & Yes \\
        \hline
        8 & Recipes & Update recipe & Yes \\
        \hline
        9 & Reviews \& Recipes & Create recipe review & Yes \\
        \hline
    \end{tabulary}
\end{table}

\subsection{Gathering Security Controls}
For CheFeed it is important that users can securely create accounts and authenticate with their credentials. Furthermore, user authorization must be configured properly. Therefore, the feature development is configured with verifications from two categories, \say{Data Storage and Privacy Requirements} and \say{Authentication and Session Management Requirements}. The summary of the sprint configuration is described in Table \ref{}. In total fourteen verifications items were generated. Although verification item \textbf{4.2} can be ignored, since CheFeed implements a stateless token-based authentication mechanism. As a result, twelve verifications items can be mapped as acceptance criteria to the proper user story.

\begin{table}[!h]
    \centering
    \caption{Authentication and authorization sprint summary}
    \label{tab:auth-sprint-summary}
    \begin{tabulary}{1.0\textwidth}{|l|L|}
        \hline
        \textbf{ID} & \textbf{Description} \\
        \hline
        \textbf{2.1} &  System credential storage facilities need to be used to store sensitive data, such as PII, user credentials or cryptographic keys \\
        \hline
        \textbf{2.2} & No sensitive data should be stored outside of the app container or system credential storage facilities \\
        \hline
        \textbf{2.3} & No sensitive data is written to application logs \\
        \hline
        \textbf{2.5} & The keyboard cache is disabled on text inputs that process sensitive data \\
        \hline
        \textbf{2.6} &  No sensitive data is exposed via IPC mechanisms \\
        \hline
        \textbf{4.1} & If the app provides users access to a remote service, some form of authentication, such as username/password authentication, is performed at the remote endpoint \\
        \hline
        \textbf{4.2} & If stateful session management is used, the remote endpoint uses randomly generated session identifiers to authenticate client requests without sending the user's credentials \\
        \textbf{4.3} & If stateless token-based authentication is used, the server provides a token that has been signed using a secure algorithm \\
        \textbf{4.4} & The remote endpoint terminates the existing session when the user logs out \\
        \textbf{4.5} & A password policy exists and is enforced at the remote endpoint \\
        \textbf{4.7} & Sessions are invalidated at the remote endpoint after a predefined period of inactivity and access tokens expire \\
        \textbf{4.12} & Authorization models should be defined and enforced at the remote endpoint \\ 
        \hline
    \end{tabulary}
\end{table}

\subsection{User Authentication and Session Management}
User authentication is an essential part for CheFeed. It allows users to add new recipes to the app. Therefore, users must be able to create a user account first. Table \ref{tab:user-registration} describes the user story and acceptance criteria for this feature. As is described, a password policy is enforced at the remote endpoint. Table \ref{tab:user-login} describes how users can login and what security verifications needs to be put in place. Finally, users need to be able to logout as described int Table \ref{tab:user-logut}.

\begin{table}[h]
    \centering
    \caption{User registration}
    \label{tab:user-registration}
    \begin{tabulary}{1.0\textwidth}{|L|L|}
        \hline
        \textbf{Description} & \textit{As a} user \textit{I can} register with a name, email and password \\
        \hline
        \textbf{Acceptance Criteria} & 
            \begin{itemize}
                \item Given that a password policy is in place at the remote endpoint which ensures that the password is at least 8 characters long, has a maximum length of 64 characters, is not silently truncated, has no character limitation
                \item No sensitive data is exposed through the user interface
                \item Keyboard cache is disabled for password input
                \item Password is not written to application logs
            \end{itemize}
        \\
        \hline
    \end{tabulary}
\end{table}

\begin{table}
    \centering
    \caption{User login}
    \label{tab:user-login}
    \begin{tabulary}{1.0\textwidth}{|l|L|}
        \hline
        \textbf{Description} & \textit{As a} user \textif{I can} login with my registered email and password \\
        \hline
        \textbf{Acceptance Criteria} & 
        \begin{itemize}
            \item No sensitive data is written to application logs
            \item System credential storage facilities needs to be used to store sensitive data such as the authorization token
            \item The keyboard cache is disabled on text inputs
            \item No sensitive data is exposed via IPC mechanesims
            \item Authentication is performed at the remote endpoint
            \item The server provides a token that has been signed using a secure algorithm
        \end{itemize} \\
        \hline
    \end{tabulary}
\end{table}

\begin{table}
    \centering
    \caption{User logout}
    \label{tab:user-logut}
    \begin{tabulary}{1.0\textwidth}{|l|L|}
        \hline
        \textbf{Description} & \textit{As a} user \textif{I can} logout \\
        \hline
        \textbf{Acceptance Criteria} & 
        \begin{itemize}
            \item The remote endpoint terminates the existing session when the user logs out
            \item Sessions are invalidated at the remote endpoint after a predefined period of inactivity and access tokens expire
        \end{itemize} \\
        \hline
    \end{tabulary}
\end{table}
